\chapter{Proposed Work} 
\label{ch:futurework}

% Introduce the content of this section
The objective of the mesh network deployment is to minimize the number of 
deployed mesh nodes with the network constraints. In this section we first 
describe the the motivation of frequency agility in mesh network deployment. 
Then we propose a graph-theoretic model for the network deployment problem
with the QoS constraints of an operational wireless mesh network.
%and discuss 
%the possible beamforming application in wireless network deployment. 

%\subsection{Motivation}
%\label{subsec:motivation}
%% Propagation -- Communication range
%Wireless propagation is the behavior of the signal loss characteristics 
%when wireless signals are transmitted through the wireless medium.
%The strength of the received signal depends on both the line-of-sight
%path (or lack thereof) and multiple other paths that result from 
%reflection, diffraction, and scattering from obstacles
%~\cite{andersen1995propagation}. The widely-used Friis equation 
%characterizes the received signal power $P_r$ in terms of transmit 
%power $P_t$, transmitter gain $G_t$, receiver gain $G_r$, wavelength
%$\lambda$ of the carrier frequency, distance $R$ from transmitter to 
%receiver, and path loss exponent $n$ according to~\cite{friis}:
%\begin{equation}
%\label{eq:friis}
%P_r=P_t+G_t+G_r+10n \log_{10}\left( \frac{\lambda}{4\pi R}\right)
%\end{equation}
%Here, $n$ varies according to the aforementioned environmental 
%factors with the value of two to five in typical outdoor 
%settings~\cite{rappaport}. Through the propagation model, in 
%the same environment with a constant path-loss exponent $n$, 
%lower frequency white space bands offer not only more bandwidth, 
%but also large communication range, which could potentially be 
%used to reduce the number of access points. 
%
%\begin{figure}
%%\vspace{-0.0in}
%\centering
%\includegraphics[width=84mm]{figures/com_range}
%\vspace{-0.1in}
%\caption{Communication Range of Access Points}                                                                 
%\label{fig:aprange}
%\vspace{-0.1in}
%\end{figure}
%
%Thus, access point with white space bands radios
%could expand coverage region and increase a single access point capacity. 
%However, at the same time, multiband radios application also increase the interference
%range which reduce the re-use ability in the network. 
%To further employ these technologies reducing the number of access points, 
%the trade off between more coverage area and interference 
%has to be optimized. In this work, we focus on this problem.
%Before starting design a network, we introduce 
%the network service constraints which are forced to 
%be followed to satisfy the clients in the deployment. 
%
%% Network Constraints
%Typically, the deployment of wireless access networks is subject to coverage and capacity
%constraints for a target area. Coverage is defined with respect to the ability of
%clients to connect to access points within their service area.  We use a coverage
%constraint ratio of $95\%$ in this work for a target area~\cite{robinson2010deploying}.
%Capacity is defined with respect to the ability of a network to serve the traffic 
%demand of clients.  Spatial reuse allows improved capacity, but increases the cost
%of deploying a network by increasing the total number of access points required.
%Hence, for densely populated areas the greatest level of spatial reuse 
%is often desired which could be offered through an expensive new access point working 
%in higher frequency.
%In contrast, sparsely-populated rural areas have lower traffic demand per unit area. Thus, 
%aggregating this demand with lower-frequency, white space bands 
%could be highly effective in reducing the total number of access points required to achieve 
%similar coverage and capacity constraints. 
%Moreover, since less TV channels tend to be occupied in sparsely 
%populated areas~\cite{msdatabase}, a larger number 
% of white space bands can be leveraged in these areas. 
%
%% Lower frequency in sparse, higher frequency in dense.  
%Under these constraints, the performance of the technology varies from dense 
%populated area to sparse area. In dense populated area, more traffic demands
%from unit area, which request more spacial reuse from higher frequency.
%In \cite{cuimeasurement}, the channel occupancy varies with population density.
%With a proper sweep schedule, more time spent for the dense part could 
%compensate the capacity occupied by other devices. 
%In sparse areas, few user generate low level traffic demand,
%with less benefit for the service. Under these conditions, an access point
%with lower frequency would be an affordable option. 
%We will model these factors as parameters in a link graph 
%and continue to analyze their influence in wireless network deployment.

\section{Model and Problem Formulation}
\label{subsec:futureproblem}

%% Assumptions of the network
%As opposed to previous works such as~\cite{franklin2007node,robinson2010deploying,si2010overview}, 
%we focus on heterogeneous multiband access point selection 
%for wireless access networks which jointly employ white space bands and WiFi bands.
%Through white space and WiFi bands frequency agility, an access point performance could be
%improve by the coverage area and throughput. 
%
%
%We assume the service vendor has a limited number of spectrum resources and 
%wireless radios have similar configuration, such as transmit power, 
%gains. Each radio on an access point operates with a classic protocol 
%model~\cite{gupta2000capacity}. Then we can further analyze the performance 
%of access point under different traffic demand distribution according to the capacity 
%and coverage constraint.
%
%% Capacity constraint
%A network deployment should ideally provide network capacity equal to the demand of the service 
%area to maintain the capacity constraint. The demand of a service area could be calculated as the 
%summation of individual demands all over the service area $D_a=\sum_{p\in P}D_p$. Since 
%household demand for Internet has been previously characterized~\cite{rosston2011household}, 
%$D_a$ could represent the population distribution $f$ and service area $k$ as 
%$D_a=\sum_{f \in F,k \in K}\bar{D_p}*f*k$. 
%The capacity constraint could be represented with access points set $M$ according to:
%\begin{equation}
%\label{eq:nlbound}
%\sum_{m \in M}C_r^m \ge \sum_{f \in F,k \in K}\bar{D_p}*f*k
%\end{equation}
%% Coverage constraint
%At the same time, the wireless network must additionally satisfy the coverage constraint in the service 
%area where the access points provide connectivity for client devices. 
%Generally, a coverage of $95\%$ is acceptable for wireless access networks~\cite{robinson2010deploying}.
%The object of this work is to find the best possible number of access points so that the network has good 
%connectivity and enough capacity to satisfy the traffic demands.
%
%% Single  access point service analysis
%Under the capacity and coverage constraints, the service area of an access point 
%is limited by the propagation range and access point capacity. 
%The radius of service area $r_s$ could be represented as:
%\begin{equation}
%\label{eq:servicearea}
%r_s=min\{r_p,r_c\}
%\end{equation}
%$r_p$ represents the propagation range of the access point, $r_c$ is the capacity range of 
%a radio in the access point. A simple example is when the traffic demand is distributed uniform in a circle, from 
%Eq.~\ref{eq:nlbound} the capacity range $r_c$ could be noted as $r_c=\sqrt{k/\pi}$. 
%Moreover, the propagation range and capacity rang could be determined by the environment, traffic distribution and
%power control~\cite{robinson2010deploying}. 
%These parameters could be pre-detected from existing measurements, census and public or private database.
%When a target area is given, we could model the traffic demand, access points and
%potential connectivity links as a graph according to the parameters from database.


% Problem
In our previous work, we address the link communication spectrum adaptation, 
single hop access network deployment, and the multihop access network channel
assignment. Next, we are going to answer the question how to locate the mesh
network access nodes and gateways in multiband spectrum adaptation scenario.
The input of the problem is a target service area given with the parameters, 
such as population distribution, spectrum usability, and FCC licensed channels, 
etc. The output is to locate the network infrastructure from the candidates.
Thus, the target area with pre-defined parameters could be modeled as a 
connectivity graph with vertices represented as the centralized traffic demand 
of a certain area and potential access points locations. The edges denote the 
links between the locations. Oppose to previous works,
~\cite{robinson2010deploying,franklin2007node,tang2005interference,irwin2013resource}
we formulate the input connectivity graph as a graph $G = (V,E,F)$, where centralized
traffic demand, access points location candidates and links from a type of access 
point defined by its frequency form a unified connectivity graph. 
We are searching for the output of the problem which is expected to be an graph 
$G' = (V',E',F')$ marks the mesh, gateway nodes and chosen frequency for them. 

The vertices in the modeled input graph represent a set $C$ of  separated target 
area with traffic demands. The set $C$ consists of physical coordinates representing 
target areas where client coverage is desired, analogous to the area to be covered 
in a geometric formulation and the traffic amount need to be served. And also the 
set of potential access points $M$ is a second part of the vertices in the modeled 
graph. The potential locations of access points are assumed known through the 
infrastructure conditions. The vertex set of the input connectivity graph is 
the union of potential access points and centralized traffic demand locations as 
$V = C\cup M$.

The access points types set $F$ is defined by the working frequency bands. It is 
a set of different combinations of frequency bands. The set $E$ in the graph is the 
physical link under protocol model between two vertices according to the access 
point type.

In the output graph, $V'$ includes the chosen access points set $M'$ and served 
traffic demand location set $C'$. The set $F'$ tells the chosen access point 
type of each $M'$. 
%The connectivity and capacity constraints could be defined 
%by the output graph $G'$, as shown in~\ref{eq:graph_coverage} and~\ref{eq:graph_capacity} 

%\begin{equation}
%\label{eq:graph_coverage}
%\frac{\sum{Number\{C'\}}}{\sum{Number\{C\}}} \ge \theta_{coverage} 
%\end{equation}
%$\theta_{coverage}$ is the desired level of coverage for the target area. $C'$ is the 
%served traffic demand location of the target area. 
% 
%\begin{equation}
%\label{eq:graph_capacity}
%C'_n \ge C_n\cdot \theta_{capacity}, C'_n \in C', C_n \in C
%\end{equation}
%$\theta_{capacity}$ is the percentage of satisfied traffic demand for the target area,
%which also include the fairness request in the equation.

The output of the graph could be optimized in several aspects. From the view of network carriers,
the number of access points would be the primary concern $Min{\{Number\{M'\}\}}$.
Through the carriers monthly income from flow charge, maximize the served traffic
demand would be the objective $Max{\{\sum{C'}\}}$. 

To optimize these parameters, we are working on a graph or game theoretic approach
