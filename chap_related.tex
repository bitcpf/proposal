\section{Related Work}
\label{sec:related}

% Deployment problem 
Wireless mesh network deployment is to design wireless architecture
for offering Internet service in an target area.
There are significant challenges in wireless mesh network deployment,
such as user priorities, user behaviors, long term throughput estimation, selfish clients,
interference and energy efficiency, etc.~\cite{tragos2013spectrum}
These challenges are distinguished under the topics of channel assignment,
cognitive radio, protocol design, etc.~\cite{tragos2013spectrum,akyildiz2006next}
Previous works have recognize the impact of interference in 
wireless mesh network deployment is the key issue~\cite{tang2005interference,irwin2013resource,chieochan2013channel}.
To overcome the challenges, a lot of works have been done to optimize the 
deployment in increasing throughput, minimize resource, reducing interference,
etc.~\cite{irwin2013resource,subramanian2008minimum,doraghinejad2014channel}
Many works have studied the network deployment problem in multihop wireless
networks~\cite{jain2005impact,akyildiz2005wireless,raniwala2004centralized,tragos2013spectrum}.
In addition, multiple radios have been used to improve the routing in multi-channel
scenarios~\cite{draves2004routing,irwin2013resource}. 
Both static and dynamic network deployments have been discussed in previous works under
the 802.11 WiFi scenario~\cite{wu2006analysis,ramachandran2006interference,subramanian2008minimum}. 
However, all of the aforementioned works have not considered propagation 
differences of the diverse frequency bands of white space and WiFi, which we show are 
critical improving the performance of mesh networks.
Frequency agility in multiband scenario brings more traffic capacity to 
wireless network deployment as well as more complexity of resolving the interference issues.

% Network deployment focuses, metrics/targets
Previous work~\cite{pcuiwinmee} involve the inter-network interference in
multiband scenario, but did not offer the solution of intra-network interference.
As a new designed wireless network, intra-network interference is 
more important for performance estimation. Previous work focus on
WiFi wireless networks proposed several methods to reduce the 
interference targeting on multiple metrics.
~\cite{tang2005interference,he2008optimizing,robinson2010deploying}
focus on reducing the gateway mesh nodes. ~\cite{irwin2013resource,subramanian2008minimum} try to reduce the
overall interference in the worst case of traffic independent scenario.
~\cite{chieochan2013channel,li2012channel} improve the performance
in throughput. However, these works fails to involve the traffic demands
of clients in their solutions.~\cite{robinson2010deploying,long2013fair} consider
the QoS requirements in the WiFi network design. Our work also
consider the traffic demands from the client as part of our 
network design to satisfy both customers and vendors.



% Solution methods
The wireless network deployment problem has been 
proved as a NP-hard problem~\cite{doraghinejad2014channel}. 
Several works introduce relaxed linear program formulation 
to find the optimization of multihop wireless networks 
\cite{tang2005interference,irwin2013resource,filippini2013new}.  
Also, game theory methods is another option to solve 
the problem~\cite{raniwala2005architecture,
wang2010game}.  
Social network analysis is also popular in wireless 
network design~\cite{zhu2013survey}.
In this work, we model the problem to a linear program
to approach the optimal solution and generate heuristic
algorithms to find a practical solution for the problem.

% White Space
To be used effectively, white space bands must ensure that available TV bands
exist but no interference exists between microphones and other devices~\cite{bahl2009white}. 
White space bands availability has to be known in prior of network deployment.
TV channels freed by FCC are fairly static in their channel assignment, 
databases have been used to account for white space channel availability 
(e.g., Microsoft's White Space Database~\cite{msdatabase}).
In fact, Google has even visualized the licensed white space channels 
in US cities with an API for research and commercial use~\cite{googledatabase}.
In contrast, we study the performance of mesh networks with a varying number 
of available white space channels at varying population densities, assuming 
such white space databases and mechanisms are in place. As FCC release these 
bands for research, many methods have been proposed to employ these frequency bands.
~\cite{bahl2009white} introduce WiFi like white space link implementation on USRP and 
link protocols. ~\cite{cui2013leveraging} discuss the point to point communication
in multiband scenario. In~\cite{filippini2013new}, white space band application is 
discussed in cognitive radio network for reducing maintenance cost. 
In~\cite{deb2009dynamic}, the white space is proposed to increase the 
data rates through spectrum allocation. 
In our work, we focus on maximizing the throughput of the wireless network.





