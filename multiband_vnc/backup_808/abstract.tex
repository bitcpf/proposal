\begin{abstract} 

The emergence of the reapportioned white space bands for data usage will fuel the already
increasing frequency band diversity of today's wireless hardware. As a result, many
existing multi-channel and/or multi-band adaptation schemes will become more important.
Many of these works have considered the capacity and channel activity of these channels,
but have not distinguished between the propagation characteristics that occur when considering
a difference in frequency of multiple GHz and the resulting effect in various environments.
In this paper, we leverage the contextual information and propagation diversity to enable
adaptation across a number of frequency bands from 700 MHz to 5.8 GHz.  To do so, we
perform a number of experiments in a vehicular environment using a campus bus.  With a
model based on a Support Vector Machine (SVM) and an in-situ training set, we can predict
the throughput on a free channel.  We can then consider the activity level per band to 
compute the net throughput we should expect on a given band to guide our adaptation protocol.  
In the field, we exploit the propagation differences experience per band to show that training on a repeatable 
route can yield vast performance improvements from prior schemes.  We show that minimal  
amounts of training can provide such improvements and that a simple scheme that can allow
multiband adaptation gains when there is insufficient levels of training.


%Probably at this time we could not employ the bus doing the experiments.

%Unused spectrum whitespaces in the currently underutilized analog TV bands are able to exploit for future wireless networks. There are potential room for performance improvement of wireless communication in throughput, power consumption and link fairness extending wireless to these bands. Previous methods are focused on channel adaptation across multiple channels in one band without considering the propogation and other characters among different bands. In this work, we employ the propogation difference for performance prediction of multiband adaptation. To identify the crowded level,we involve an activity level of networks based on the statistics information during a time slot to make the prediction more accuracy. The amount of context information required for multiband adaptation and the influerence of window size for activity level are evaluated in this paper. We conduct indoor and in-field experiments to validate our method. The experimental results demonstrate that our method is able to achieve as ... 


\end{abstract}


