\section{Experiment Design}
\label{sec:experiment design}



In this section, we use emulated channel for repeatability and control to directly compare and evaluate different bands performance and design in-field experiments to have the data for verification on campus. The experimental results indicate that the application of multi-band adaptation can enchance the performance of a transmitter and receiver pair. 
Fixme%discribe the results 

%context information experiments
\subsection{Emulated Channels Experiments}
In order to collect data for creating context information database, we use an experimental setup where two wireless nodes communicate across emulated channels. The channel emulator Azimuth ACE-MX is used for channel emulation, allowing controllable propagation, fading characteristics, velocity and other parameters with a broad range of industry-standard models for our experiments\cite{AzimuthACE}. Azimuth ACE MX can simulate a wireless channel band across wide frequency, in 450MHz-2700MHz, 3300MHz-3800MHz, 4900MHz-5900MHz\cite{AzimuthACE}. The channel emulator can create repeatable channels for testing each wireless band to measure the performance for a given wireless context. Each mode represents a band with fixed packet size and other parameters. For a given band, we can repeat the experiment for a while to get the performance across different bands in the ideal channel environmemt.

Fixme{emulator photo}

Fixme{Should we mention RB433?}

To ensure that our results are broadly applicable across wireless device, we employ widely accepted 802.11 testbed. Gateworks 2358 with Ubiquiti XR serial radios, XR2(2.4GHz),XR5(5.8GHz),XR7(700MHz),XR9(900MHz) make up our testbed hardware\cite{Gateworks,Ubnt}. OpenWRT is installed on the testbed hardware as an open source and widely used Linux-based operating system for wireless devices\cite{Openwrt}. Through this multi-radio testbed, we have the capacity to measure the performance in the same traffic generating system, the same channel state for different bands comparison.

Fixme{Figure of emulator measurement}

The experimental setup and data flow shows in Fixme %\figure{}
The testbed receiver side captures the variation of the SNR and report the throughput values from Iperf\cite{Iperf}. Based on the data, the band adptation algorithm will extract the relationship between the contextual information and the target band.

%in-field experiments design
%Gateworks, socket, spectrum analyzer
\subsection{In-field Experiments}

To get the \emph{Activity Level,SNR} as the input of our algorithm, we need to map the total received packets, our focused transmitter-receiver pair received packet, received signal power, background noise and interference. We use two Gateworks platform with Ubiquiti radios combining Agilent Spectrum Analyzer to collect data for our off-line analysis. We map these data synchronous through timestamp from GPS information on our testbed and the system time on the Spectrum Analyzer.

\subsubsection{Activity Level Measure} 
To calculate the \emph{Activity Level}, the raw data we need is the total received packets and the focused transmitter-receiver pair received packets during a small time slot. Madwifi is a driver widely embedded in Linux based wireless operating system, such as the open source operating system OpenWRT\cite{Madwifi,Openwrt}. Madwifi could report the total received packet(RX packet) of one radio in the driver statistics.
To measure the throughput and calculate the focused transmitter-receiver pair received packet(Focused RX packet) in a less than 1 second time slot, a socket program is generated for the experiments. The transmitter side could send packets in a 500ms duration, then turn off the transmission for another 500ms to meaure the interference. These raw data could be put into \ref{equation:Activity Level} calculate the activity level for the future prediction.



\subsubsection{Signal Measure}

We need the pure signal power from the transmitter to extract the ideal throughput from the context information we got from emulated channel experiments. It is impossible to get the interference during the transmitter sending period. So the transmitter is configured to turn on for a 500ms transmission to measure the throughput and the signal power(noise, interference and transmit power), and then turn off for another 500ms to measure the interference strength. 

Fixme figure of the experiment flow

%SA utility
The testbed could report the received strength, howerve, it could not update the value without transmission. The method we use to collect the signal strength is to record the spectrum activity through an Agilent Spectrum Analyzer. Agilent MXA N9020 Signal Analyzer could monitor a wide band spectrum and record the spectrum activity in a CSV file for a time point \cite{SA}. To get the continous record during the experiments, a MFC(Microsoft Foundation Class) dialog software is generated to control the spectrum analyzer to save the record file periodicly through Agilent I/O Command Expert\cite{MFC,AC}. We also record the signal strength on the testbed to match the record from spectrum analyzer.

A node in a car works as the transmitter, and another node with the spectrum analyzer located in a fixed place work as the receiver. The car drive one loop on campus for a measurement of one band. During one loop, the throughput, RX packets, Focused RX packets, signal strength in transmission on and transmission off are collected. 


Then based on these data, the off-line data process will show the gain of multi-band adaptation.  
%

