\section{Data Process}
\label{sec:experiment}

Fixme


%Detail may not useful
Parsing script
In our experiments, we are constantly collecting large amounts of data, including the received signal level, current location, velocity, and time of day. Working with the memory-limited Gateworks boards, it became necessary to implement a solution to collect large amounts of data without exhausting the available memory space on the boards. Thus, we compiled a script to parse an undetermined number of data files containing the raw data collected from the ongoing experiments. Utilizing the Perl programming language, which the Gateworks boards are capable of running, the script scans every file in a directory we specify and parses them, looking specifically for signal level, lat/long location, velocity, and time data recorded from the experiments. Upon finding this information, the script reformats the data by placing it into a .csv file. Additionally, using the location data, the script calculates the  distance between the transmitting and receiving boards and adds this information to the .csv file. Upon parsing the data, the raw experimental data files can simply be delete from memory, freeing up space for the data of subsequent experiments.

Activity monitor script
For in-situ experiments, the need became apparent to track the number of new incoming packets and compare it with the number of previously received packets. Additionally, this needed to be done for each of the four wireless radios on the board. In doing this, we identify the most efficient frequency band to transmit data. To implement this system, a script was needed to run efficiently in the background while experiments were taking place. To achieve this, we wrote a bash shell script to run directly on the board without relying on any higher level programming language that could potentially cause greater performance overhead. As a result, the script only consumes one to two percent of CPU resource. The script begins by examining the received bytes across each radio for a length of 30 seconds and placing the bytes received each second on a new line in a file. Upon the completion of the 30 second buffering time period, the next second of received bytes on each radio is read and compared with the last 30 seconds of received data. This ratio of the most recent received data to old received data is then calculated and written to four files, one for each radio, for its subsequent use in selecting which radio to transmit/receive from.

