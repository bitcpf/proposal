% Challenge in wireless communication
Due to the convenience of deployment and utility, wireless devices and network
has become the dominant tools for telecommunication. However, the spectrum resource
is limited in natural and has to be approved for accessibility, there is a huge 
challenge for individual devices and infrastructure to efficiently exploit the 
licensed frequency. In 2009, the FCC has approved the use of broadband services 
in the white space of UHF TV bands, which were formerly exclusively licensed to 
television broadcasters. These white space bands are now available for unlicensed 
public use, offerring new opportunities for new design of device and network with 
better performance in throughput and economy cost. In this work, we investigate 
multiband adaptation in both link communication and network deployment. Furthermore, 
we design measurement driven algorithms to improve both the link and network 
performance.
% List previous papers
% VNC
First, we leverage knowledge of in-situ operation across frequency bands with 
real-time measurements of the activity level to select the the band with the 
highest throughput. To do so, we perform a number of experiments in typical 
vehicular topologies. With two models based on machine learning algorithms 
and an in-situ training set, we predict the throughput based on: ({\it i.}) 
prior performance for similar context information ({\it e.g.}, SNR, GPS, 
relative speed, and link distance), and ({\it ii.}) real-time activity level 
and relative channel quality per band. In the field, we show that training on 
a repeatable route with these machine learning techniques can yield vast 
performance improvements from prior schemes. 
% Winmee
Second, while many metropolitan areas sought to deploy city-wide WiFi networks, 
the densest urban areas were not able to broadly leverage the technology for 
large-scale Internet access.  Ultimately, the small spatial separation required 
for effective 802.11 links in these areas resulted in prohibitively large up-front 
costs. The far greater range of these lower white space carrier frequencies are 
especially critical in rural areas, where high levels of aggregation could 
dramatically lower the cost of deployment and is in direct contrast to dense 
urban areas in which networks are built to maximize spatial reuse.  Thus, 
leveraging a broad range of spectrum across diverse population densities 
becomes a critical issue for the deployment of data networks with WiFi and white 
space bands. We measure the spectrum utility in the Dallas-Fort Worth metropolitan 
and surrounding areas and propose a measurement-driven band selection framework, 
Multiband Access Point Estimation (MAPE). In particular, we study the white space 
and WiFi bands with in-field spectrum utility measurements, revealing the number 
of access points required for an area with channels in multiple bands. In doing so, 
we find that networks with white space bands reduce the number of access points by 
up to 1650\% in sparse rural areas over similar WiFi-only solutions. In more populated 
rural areas and sparse urban areas, we find an access point reduction of 660\% and 
412\%, respectively.  However, due to the heavy use of white space bands in dense 
urban areas, the cost reductions invert (an increase in required access points 
of 6\%).  Finally, we numerically analyze band combinations in typical rural and 
urban areas and show the critical factor that leads to cost reduction: considering 
the same total number of channels, as more channels are available in the white space 
bands, less access points are required for a given area.
% WM
Third, wireless mesh networks were previously thought to be an ideal solution for
large-scale Internet connectivity in metropolitan areas.  However, in-field
trials revealed that the node spacing required for WiFi propagation 
induced a prohibitive cost model for network carriers to deploy. The digitization 
of TV channels and new FCC regulations have reapportioned spectrum for data 
networks with far greater range than WiFi due to lower carrier frequencies. 
Also, channel occupancy of change the performance of wireless network across 
both ISM bands and white space bands. In this work, we consider how these white 
space bands can be leveraged in large-scale wireless mesh network deployments 
with in-field measured channel capacity. In particular, we present an integer 
linear programming model to leverage diverse propagation characteristics and 
the  channel occupancy of white space and WiFi bands to deploy optimal WhiteMesh 
networks. Since such problem is known to be NP-hard, we design a measurement driven 
heuristic algorithm, Band-based Path Selection (BPS), which we show approaches 
the performance of the optimal solution with reduced complexity.  We additionally 
compare the performance of BPS against two well-known multi-channel, multi-radio 
deployment algorithms across a range of scenarios spanning those typical for 
rural areas to urban areas. In doing so, we achieve up to 160\% traffic achieved 
gateways gain of these existing multi-channel, multi-radio algorithms, which are 
agnostic to diverse propagation characteristics across bands.  Moreover, we show 
that, with similar channel resources and bandwidth, the joint use of WiFi and 
white space bands can achieve a served user demand of 170\% that of mesh networks 
with only WiFi bands or white space bands, respectively. Further, through the 
result, we leverage the channel occupancy and spacing impacts on white mesh
network and study the general rules of band selection.
% Whiteteris
Furthermore, we model the heterogeneous white space and WiFi access tier 
deployment problem and propose a relaxed ILP to get the lower bound of the 
amount of access point under resource limitations and a heuristic approach 
of the problem. We start from the uniform population distribution wireless 
network deployment, then discuss the non-uniform population distribution 
wireless network deployment. Then, we discuss white space band application
in multiple spectrum resource degree. In particular, we map the problem as 
a Bin packing problem and resolve it with Multiband Hetegeneous Access Point 
deployment method. In doing so, we discuss the benefit of white space bands 
in reducing the number of acess points and provide heuristic solution for 
access points deployment. Our numerical simulation shows tha hetergeneous 
access point could benefit most of the scenarios in wireless access tier 
deployment.

% Future work
Lastly, we propose to study new solution for multiband application in large 
scale network deployment. Graph theory is a efficient way to model and solve 
the deployment problem. The wireless multiband network deployment is formulated
as a graph theory problem to locate the gateway nodes. In our previous work,
we focus on solve the channel assignment and access tier mesh nodes deployment
in multiband scenarios. With the propose work of graph solution of gateway nodes
location, a complete process of multiband wireless network solution could be 
built. Also we are going to discuss beamforming in wireless network deployment. 
Beamforming provides potential gains for reducing the deployment cost and improving
the fairness in wireless network deployment. We are going to exploit the great
spacial agility from beamforming for future wireless networks.

