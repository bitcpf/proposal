
We investigate the question of similarity in homogenous isotropic turbulence using the well known shell models, GOY and Sabra.  These are crude models of the 3D Navier Stokes equations in spectral form, but with far fewer degrees of freedom.  In fact, there is only one complex variable, $u_n$, for each octave (shell) in wave number space, $2^{n-1}<|\vec{k}|<2^{n}$.  The shells interact within their nearest neighbors and form a dynamical system with a strange attractor.  Our dynamical system is forced at the largest scale (smallest $n$), in a statistically steady fashion. The energy, pumped in by the forcing, cascades through the shells to the smallest scales (largest $n$) where dissipation is active.  In the intermediate scales, only inertial forces are active.  This is the range of scales that exist when the Reynolds number is high and forms part of the so-called universal equilibrium range.  We analyze the statistics of the time series of $u_n(t)$ and find that self-similarity exist between shells in the inertial range. Specifically, the pdf of $u_n$ can be described by a similarity formula involving an affine transformation on a logarithmic axis. We use data from shell model simulations to estimate parameters in a recently developed theory for the inertial range.
