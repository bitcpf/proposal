\chapter{Introduction} \label{ch:introduction}



Due to the ubquity and covenience of wireless Internet service, the demand of 
wireless device users for efficient wireless links and networks are surging.
%Wireless network is believed to be the most convenient method to get service 
%for users. The demand of wireless device users inspires vendors to provide 
%faster, stable service in their networks. Because of the prevalence of 
%individual devices, vehicular wireless devices and wireless sensors, 
For example, wireless 
data traffic has increased for a thousand-fold over the last years and is 
expected to continue over the next decade~\cite{metis}, greatly motivating 
the improvement of wireless link communication and network deployment technology.

However, unlike wired networks, wireless devices share spectrum resources 
and have unstable, time-varying channel quality. For a vehicular
environment, the situation is even worse due to the frequently changing channel 
state. Drivers and passengers around the world could utilize a 
wide array of vehicular applications ranging from real-time traffic 
monitoring and safety applications to various infotainment applications.
However, the continuous use of such applications is limited due to the
challenge of transmitting over highly-dynamic vehicular wireless channels.
In such networks, the increasing availability of different frequency bands 
with correspondingly diverse propagation characteristics could allow flexibility 
and robustness of vehicular links. Even with spectral flexibility, links are 
extremely tenuous, demanding instantaneous decisions to remain connected, 
motivating an algorithm that can find the appropriate frequency band quickly 
and according to the current environmental context.

Clearly, more available spectrum could improve the performance for 
both links and networks. The FCC has approved the use of broadband 
services in the white spaces of UHF TV bands, which were formerly exclusively 
licensed to television broadcasters. These white space bands are now available 
for unlicensed public use, enabling the deployment of wireless access networks 
across a broad range of scenarios from sparse rural areas (one of the key 
applications identified by the FCC) to dense urban areas~\cite{carlson}. The 
white space bands operate in available channels from 54-806 MHz, having a far 
greater propagation range than WiFi bands for similar transmission power
~\cite{balanis2012antenna}. In WiFi and white space heterogeneous wireless 
network, the service area degree of an access point depends on the capacity 
of radios, the propagation range and the demands of the serving area. The scant 
frequencies of radios, the propagation distinctive and the demands diversity 
of population distribution bring the variation of an access point service area. 
These issues are substantial to designing an optimal network deployment and 
provide potential commercial wireless services to clients in any location.

White space frequencies is not only benefit the access tier network, 
but also provides opportunity for backhual tier networks. About a 
decade ago, numerous cities solicited proposals from network carriers 
for exclusive rights to deploy city-wide WiFi, spanning hundreds of 
square miles. While the vast majority of the resulting awarded 
contracts used a wireless mesh topology, initial field tests revealed 
that the actual WiFi propagation could not achieve the proposed mesh node
spacing. As a result, many network carriers opted to pay millions of 
dollars in penalties rather than face the exponentially-increasing
deployment costs (e.g., Houston~\cite{cnet_aug07} and Philadelphia
~\cite{arstechnica_may08}). Thus, while a few mesh networks have been 
deployed in certain communities~\cite{CRSK06,google_imc08}, wireless 
mesh networks have largely been unsuccessful in achieving the scale 
of what was once anticipated~\cite{taps}. Around the same time, the 
digital TV transition created more spectrum for use with data networks
~\cite{fccwhitespace}. These white space bands operate in available 
channels from 54-806 MHz, having increased propagation characteristics 
as compared to WiFi~\cite{balanis2012antenna}. Hence, the FCC has 
identified rural areas as a key application for white space networks 
since the reduced population from major metropolitan areas allows a 
greater service area per backhaul device without saturating wireless 
capacity. Naturally, the question arises for these rural communities 
as well as more dense urban settings: how can the emerging white space 
bands improve large-scale mesh network deployments?  While much work 
has been done on deploying multihop wireless networks with multiple 
channels and radios, the differences in propagation have not be 
exploited in their models~\cite{raniwala2004centralized,tang2005interference, si2010overview}, 
which could be the fundamental issue for the success of mesh networks 
going forward.

As white space frequencies are made available for wireless communication,
both the academic and  industry experts are seeking an integrated approach
for white space and WiFi bands in wireless links and networks. 
In this work, we seek to jointly use 
white space bands and WiFi bands in vehicular and large scale network deployments. 
We also
propose prospective graph theory solution for white space spectrum 
in short future. 
%Beamforming is recognized as an prospective 
%technology to further improve the performance of wireless networks, 
%through increasing both time reuse and spatial reuse. The potential 
%gains of reducing deployment cost and fairness from beamforming 
%motivated us to exploit the jointly usability of beamforming and 
%white space bands in the future.


\section{Challenges}
% Challenges

Wireless links and networks suffer from the limited available resources
and interference. Everyone must 
share the bandwidth. Here, users and work carriers alike seek 
better service and lower cost for wireless networks. 
Spectrum agility offers a new promising resource to address these issues. 
%the spectrum agility, the diversity of spectrum and network 
%complexity have to be recognized prior to the solutions.

% VNC challenge
First, accurate knowledge of channel state is challenging to obtain due to the
time varying nature of channel quality and interference.
at the receiver. In recently, cognitive radio mechanisms which 
interleave channel accesses motivate the frequency band selection 
problem of finding the optimal spectrum on which to transmit
~\cite{ghasemi2008spectrum}. Prior work has considered a number of 
challenges in leveraging white space frequencies including spectrum 
sensing, frequency-agile operation, geolocation, solving stringent 
spectral mask requirements, and providing reliable service in unlicensed 
and dynamically changing spectrum~\cite{shellhammer2009technical}. In 
particular, there has recently been an acceleration in spectrum sensing 
work~\cite{rayanchu2011fluid, kim1996pulse,cabric2004implementation}. 
Based on these works, protocols have been built for multi-channel and/or 
multiband wireless operation~\cite{MOAR,raychaudhuri2003spectrum,sabharwal2007opportunistic}. 
Other works have presented methods for searching for the most efficient 
transmission channel~\cite{mo2005comparison}, discovering channel 
information~\cite{rayanchu2011fluid, sabharwal2007opportunistic}, and 
estimating channel quality~\cite{MOAR}. Moreover, the emergence of a 
number of diverse sensors on a vehicle motivates work on heterogeneous 
wireless networks, which have different frequency bands and technologies
~\cite{hossain2010vehicular}. Thus, the various communication standards 
have diverse throughput capacity, allowing the choice of technology 
to possibly usurp frequency band decisions. For example, an 802.11n 
link at 5.8 GHz with high levels of loss might still be a better choice 
than a Bluetooth link at 2.4 GHz with little loss due to the discrepancy 
of hundreds of Mbps in throughput capacity. While these works have 
considered spectral activity and developing protocols and algorithms to 
find spectral holes, less of a focus has been on coupling such information 
with historical performance in a given propagation environment.

Second, in access layer network deployment, specific to rural areas, 
the lack of user density and corresponding traffic demand per unit area 
as compared to dense urban areas allows greater levels of spatial 
aggregation to reduce the total number of required access points, 
lowering network deployment costs. In densely populated urban areas, 
the greater concentration of users and higher levels of traffic demand 
can be served by maximizing the spatial reuse. While many works have 
worked to address multihop wireless network deployment in terms of 
maximizing served user demand and/or minimizing network costs, the 
unique propagation characteristics and the interference from coexisting
activities in white space bands have either not been jointly studied 
or assumed to have certain characteristics without explicit measurement
~\cite{si2010overview}. Specifically, previous work has investigated 
wireless network deployment in terms of gateway placement, channel 
assignment, and routing~\cite{he2008optimizing,marina2010topology}.
However, each of these works focus on the deployment in WiFi bands 
without considering the white space bands. Moreover, the assumption 
of idle channels held in these models fails to match the in-field 
spectrum utility, which could degrade the performance of a wireless 
network. These two issues are critical for designing an optimal 
network deployment and providing commercial wireless services to 
clients in any location.

Third, in the backhual layer, researchers have done both centralized and 
distributed work for channel assignment and routing~\cite{raniwala2004centralized,wu2006distributed}.
However, the FCC's new policy bring new opportunities
and challenges for these issues.  Increased channel capacity and 
propagation diversity forces new consideration for existing solutions.
Since even the original multichannel problem has been proved to be NP-hard
which could not be solved in polynomial time, low complexity
solution for joint white space and WiFi usage is needed.
%the new white space application is an emergency for 
%both academy and industry. 
As shown in Google database~\cite{googledatabase}, 
the additional number of available white space frequency channels 
vary from city to city in US. Existing channel occupancy discussed in 
previous work~\cite{pcuiwinmee} has shown that the occupancy of 
frequency impacts on wireless network deployment. Naturally, the 
question arises for improving the performance as well as the optimization 
of utilization: {\it how can the emerging white space bands improve 
large-scale mesh network deployments?}  While much work has been done 
on deploying multihop wireless networks with multiple channels and 
radios, the differences in propagation have not been exploited in their 
models~\cite{tang2005interference, long2013fair,doraghinejad2014channel}, 
which could be {\it the} fundamental issue for the success of mesh 
networks going forward.

Moreover, the joint use of WiFi and white spaces  in 
access points depends on the capacity of radios, the propagation range 
and the demands of the service area. 
Thus, the new opportunities created by white spaces motivate the following 
questions for wireless Internet carriers, which have yet to be addressed: 
{\it (i) To what degree can white space bands reduce the network deployment 
cost of sparsely populated rural areas as opposed to comparable WiFi-only 
solutions?} and {\it (ii) To what degree can heterogeneous access points 
benefit the dense population areas and sparsely populated rural areas?}

%Lastly, beamforming make the wireless network with more spatial reuse 
%similar to the wired network in some degree. However, lacking of protocols 
%and assignment methods make it hard to apply the technology in practice. 
%The issues of beamforming in application, such as deafness, new hidden 
%problem need to be leveraged and resolved by new protocols and algorithms. 
%This new technology inspire the optimization more challenging and meaningful.

% Contributions of previous work and future work plan
\section{Contributions and Future Work}

This work proposes spectrum adaptation for both the wireless links and 
large scale network deployments to improve the performance of wireless devices in 
multiple frequency bands.

% VNC Contributions
For link adaptation, we develop multiband protocols 
which couple the prior knowledge of in-situ performance of various bands 
with the instantaneous knowledge of spectral activity, SNR, and the current 
location of each band to arrive at a decision for the optimal band on which to transmit. 
To do so, we use an off-the-shelf platform that allows direct comparison and 
simultaneous experimentation across four different wireless frequency bands 
from 450 MHz to 5.8 GHz with the same physical and media access layers. 
We first develop a framework for multiband adaptation using both historical 
information and instantaneous measurements. This framework is broad enough 
to study adaptation across licensed and unlicensed bands, including white 
space frequency bands. We propose two different machine-learning-based 
multiband adaptation algorithms. The first machine learning algorithm, 
referred to as the Location-based Look-up Algorithm, is based on the idea 
of $k$-nearest-neighbor classification. The second machine-learning-based 
algorithm uses decision trees for classification. For comparison, we also 
create two baseline adaptation algorithms which attempt to make the optimal
band selection based on only: (i.)~historical performance data, and (ii.)
~instantaneous SNR measurements across various bands. We perform extensive 
outdoor V-2-V experiments to evaluate the proposed algorithms. Our results 
indicate that the proposed machine learning based algorithms improve throughput 
by up to $49.3\%$ over these baseline methods.

% WINMEE
In the access layer network deployment, we perform a measurement study 
which considers the propagation characteristics and observed in-field 
spectrum availability of white space and WiFi channels to find the total 
number of access points required to serve a given user demand. Across 
varying population densities in representative rural and metropolitan 
areas, we compare the cost savings (defined in terms of number of access 
points reduced) when white space bands are not used. To do so, we first 
define the metric to quantify the spectrum utility in a given measurement 
location. With the in-field measured spectrum utility data in metropolitan 
and surrounding areas of Dallas-Fort Worth (DFW), we calculate the 
activity level in WiFi and white space bands. Second, we propose a 
measurement-driven framework to find the number of access points required 
for areas with differing population densities according to our measurement 
locations and census data. We then evaluate our measurement-driven framework, 
showing the band selection across downtown, residential and university 
settings in urban and rural areas and analyze the impact of white space 
and WiFi channel combinations on a wireless deployment in these 
representative scenarios. 
% Paper contributions
The main contributions of our work in this part are as follows: We perform in-field 
measurements of spectrum utilization in various representative scenarios 
across the DFW metroplex, ranging from sparse rural to dense urban areas 
and consider the environmental setting (e.g., downtown, residential, or 
university campus). We develop a measurement-driven Multi-band Access 
Point Estimation (MAPE) framework to jointly leverage propagation and 
spectrum availability of white space and WiFi bands for wireless access 
networks across settings. We analyze our framework under capacity and 
coverage constraints to show that, with white space bands, the number 
of access points can be greatly reduced from WiFi-only deployments by 
up to 1650\% in rural areas. We quantify the impact of white space and 
WiFi channel combinations to understand the trade offs involved in choosing 
the optimal channel setting, given a certain number of available channels 
from multiple bands. Furthermore, we discuss to centralized using frequencies 
or distributed using them in single access point. Through our numerical 
simulation, we show that in dense area centralized using spectrum gain 
more than sparse area.
% MOA
For the heterogeneous access tier optimization, we perform a relaxed linear 
program which considers the variation of heterogeneous access point service 
area to find the lower bound total number of access points required to 
serve a given user demand. Further, we represent an Multiband Heterogeneous 
Access Point Deployment (MHAPD) greedy algorithm 
to approach the lower bound. Across varying white space and WiFi radios 
combination, population densities in representative rural and metropolitan 
areas we compare the cost savings (defined in terms of number of access 
points reduced) when white space bands are not used. We then evaluate our 
MHAPD algorithm, showing the heterogeneous band selection across downtown, 
residential and university settings in urban area and rural areas and analyze 
the impact of white space and WiFi combinations on a wireless deployment in 
these representative scenarios.
% Paper contributions
The main contributions of our work in this topic include: We develop an 
optimization framework based on linear programming to jointly leverage 
white space and WiFi bands approaching the lower bound in terms of the number 
of access points to serve the demands of a given area. We design a MHAPD 
greedy algorithm, which models the problem as a bin packing problem. We 
evaluate the performance of the presented algorithm, comparing with the 
lower bound and the hexagon of WiFi access point deployment in sparse rural 
areas given similar channel resources. The numeric results shows that our 
algorithm gains 260\% against the linear hexagon of deployments.  We 
further analyze the performance of heterogeneous access point performance 
across variety of population densities. The numeric results show that heterogeneous 
access points which jointly use white space and WiFi bands could improve 
the budget saving up to 323\%. 

% Whitemesh
In the backhual layer network deployment, we leverage the diversity 
in propagation of white space and WiFi bands in the planning and deployment
of large-scale wireless mesh networks. To do so, we first form an 
integer linear program to jointly exploit white space and WiFi bands 
for optimal WhiteMesh topologies. Second, since similar problem 
formulations have been proved to be NP-hard~\cite{jain2005impact}, 
we design a measurement driven heuristic algorithm, Band-based Path 
Selection (BPS). We then apply the approaching method in multiple 
scenarios with in-field measurement data. Across a wide range of 
scenarios, including network size, population distribution, deployment 
distance gap, we exploit when and  how to emerge white space 
bands in mesh networks. The performance of our scheme is compared 
against two well-known multi-channel, multi-radio channel assignment 
algorithms across these scenarios, including those typical for rural 
areas as well as urban settings. We further discuss how the channel occupancy 
impacts on wireless networks and show the comparison of our algorithm 
and previous methods in typical scenarios. Finally, we quantify the degree 
to which the joint use of both band types can improve the performance of 
wireless mesh networks. We analyze the white space bands application in 
wireless network deployment and develop an optimization framework based 
on integer linear programming to jointly leverage white space and WiFi 
bands to advantages and disadvantages in wireless mesh networks with measured
channel occupancy. We build a heuristic measurement driven algorithm, 
Band-based Path Selection (BPS), which considers the diverse propagation, 
overall interference level of WiFi and white space bands with measurement 
adjust. We perform extensive analysis across offered loads, network sizes, 
mesh nodes spacing and WiFi/white space band combinations, to compare 
against previous multichannel multiradio algorithms. 
We discuss the channel occupancy and mesh spacing impacts the 
performance of similar channel resources (bandwidth and transmission 
power), We show the improvement of our BPS algorithm in typical configurations up to 
180\% versus. previous multichannel algorithms.



% Future work
In the future, we propose to find a better solution based on graph
theory for multiband wireless network deployments. Future work will
provide a mechamism for not only channel assignment, but also where 
we should locate 
the gateway nodes with a less complex solution to achieve a 
complete framework for multiband wireless network deployment. 
%Also 
%we are interested in involving beamforming to increase the spatial
%reuse in wireless network with more capacity and fairness. 
%In the beamforming application, we are going to design protocols to 
%optimize the network deployment cost under the limitation of QoS.
%
% Beamforming network time? maybe could be a concept similar to network efficiency


% Thesis Overview
\section{Proposal Overview}

In Chapter~\ref{ch:background}, we discuss the background of this work, 
including the basic knowledge related to this work and the hardware 
platform and software tools applied in our research. Chapter~\ref{ch:vnc} 
provides an analysis and solution of vehicular link adaptation across multiple frequency bands. 
We also propose an access network deployment 
framework in Chapter~\ref{ch:winmee} and discuss the optimization of a 
single access point with spectrum limitations. Then, we further
compare this solution to one which both white space bands and WiFi 
bands to quantify the benefit of white spaces. In Chapter~\ref{ch:wm}, we propose an 
algorithm to assign channels in multiband wireless network scenarios for 
optimizing the throughput performance in multihop backhual networks. Further, we 
quantify the heterogeneous application of white space bands and WiFi bands 
in access points deployment process. In Chapter~\ref{ch:futurework}, we 
propose the future in multiband network deployment using traph theory for locating
gateway nodes. 
Finally, we conclude our work in Chapter~\ref{ch:conclusion}.

