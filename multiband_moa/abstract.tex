\begin{abstract}
While many metropolitan areas have plan to deploy city-wide WiFi networks, 
the densest urban areas where not able to broadly leverage the technology 
for large-scale Internet access within limited budget.
Ultimately, the small spatial separation required for effective 802.11 links 
in the areas resulted in prohibitively large upfront costs.
% FCC release white/gray space bands
% Bands difference
The FCC has reapportioned spectrum from TV white spaces for the purposes of 
large-scale Internet connectivity via wireless topologies of all kinds. 
The far greater range of these lower carrier frequencies are especially critical 
in rural areas, where high levels of aggregation could dramatically lower the 
cost of deployment and is in direct contrast to dense urban areas, in which 
networks are built to maximize spatial reuse. 
Thus, leveraging heterogeneous structure of spectrum across diverse population densities becomes  
a critical issue for the deployment of data networks with WiFi and white space bands.
% What we did
% Gains
In this paper, we model the heterogeneous white space and WiFi access tier 
deployment problem. We propose a relaxed ILP to get the lower bound of the 
amount of access point under resource limitations and a heuristic approach 
of the problem. We start from the uniform population distribution wireless 
network deployment, then discuss the non-uniform population distribution 
wireless network deployment. Furthermore, we discuss white space band application
in multiple spectrum resource degree. In particular, we map the problem as a Bin packing problem 
and resolve it with a fixme method. In doing so, we discuss the benefit of 
white space bands in reducing the number of acess points and provide heuristic
solution for access point selection. Our numerical simulation shows that
hetergeneous access point could benefit most of the scenarios in wireless
access tier deployment.
\end{abstract}
