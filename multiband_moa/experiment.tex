\section{Numerical Evaluation and Analysis}
\label{sec:moaexperimentdesign}

To evaluate the performance of heterogeneous access point wireless network deployments,
we perform a numerical evaluation with the linear program, and the MHAPD algorithm to analyze the role
of white space and WiFi bands in terms of total access points required for a given target 
area.

\subsection{Experimental Setup}

% 4 bands, capacity, individual traffic demand,  
In the evaluation, we set the demand request as 2 Mbps per person with the population
density from 20 to 2000 per square kilometer. We assume $20\%$ residents will use this
service, the maximum transmit power is 30 dBm, and a path loss exponent of $3.5$ from 
Ref.~\cite{meikle2012global}. 

% Channel availability and influence, and in hexagon model
We adopt the 802.11n maximum data rate of 600 Mbps. In the protocol model, the interference 
range is as twice as the communication range. We investigate both traffic demand and the 
number of white space channel influence on heterogeneous wireless network deployment. 
We have interference free scenario, each band has at least 3 channels, which is available for
most rural areas and some cities, such as Houston~\cite{googledatabase}. In this scenario,
it is possible to use all heterogeneous access points since there are enough frequency resources
in the area. However, in the field, there are some cities has area only one or two licensed 
white space channels, such as Salt Lake City~\cite{googledatabase}. In these cities, 
only part of the access points could be heterogeneous due to the limited frequency resources. 
%We run numerical simulation of both the scenarios and analyze the heterogeneous access points 
%amount from the results.

We given the target area as $15\times 15$ square kilometers. In the numerical simulation, 
we assign orthogonal WiFi channels in 2.4 GHz,5.8 GHz and white space channels in 450 MHz, 
800MHz. Then we calculate the service area of access point according to their radio 
combinations as described in~\ref{subsec:problem} with a hexagonal model. Then we 
run our linear program and the MHAPD algorithm to investigate the benefit from white space band 
and in what degree heterogeneous access points is beater than single radio access point. 

\subsection{Results and Analysis} 
\label{subsec:result}

% Results
Fig.~\ref{fig:enoughchannels} shows the access points number for serving the target area with 
more than 3 white space channels, the scenario that white space radios could be used on all 
the access points in hexagonal deployment model. In the simulation, we set 3 white space channels
in 450 MHz. As shown in Fig.~\ref{fig:enoughchannels}, at the beginning, the served area of WiFi 
only access point is restricted by the communication range. As the population distribution increase, 
the served area of WiFi only access point will be limited by the capacity constraint instead 
of the communication range which make the service area of an access point smaller. The curve of 
hexagonal WiFi deployments keeps flat until the traffic demand becomes the limitation of the service
area. In the heterogeneous deployment, the served area is restricted by the traffic demand at 
the beginning, the number of access point increase as the traffic demand increase. Also since 
there are enough channels can be reused, our algorithm use almost the same number of access point 
as the linear program to serve the target area. In this scenario, as population distribution 
increase, the gain of adding white space channels in our algorithm comparing with WiFi only(2.4 GHz) 
decrease from 686\% to 248\% and keep around the value 260\%. The gain is generated from the large 
propagation range of white space bands at low population distribution. As the population distribution 
increase, which means the traffic demand increase, some of the gain comes from more capacity of 
heterogeneous access point.


\begin{figure}
%\vspace{-0.0in}
\centering
\includegraphics[width=74mm]{figures/enoughchannels}
\vspace{-0.1in}
\caption{Sufficient White Space Channels Scenario}                                                                 
\label{fig:enoughchannels}
\vspace{-0.1in}
\end{figure}

With few white space band chanels(less than 3 channels), in the hexagonal model, only one or 
two neighbor access points could use white space channels. Fig.~\ref{fig:onewhitechannel}
shows the simulation result of one 450 MHz channel deployments. Shortage of white space channels
make the number of access points much more than the linear program calculation. From the 
simulation, the gain of white space channel application achieves between 323\% and 86\%
over the hexagonal deployment. 

In these sufficient and shortage of white space channel scenarios, the highest gains are achieved with 
low population distribution. This result represent that white space band fits rural or suburban
area where has few residents better than dense area. When the population distribution increase, the 
benefit of white space channels generated by the bandwidth, which also could be implemented by adding 
more WiFi channels.

\begin{figure}
%\vspace{-0.0in}
\centering
\includegraphics[width=74mm]{figures/onewhitechannel}
\vspace{-0.1in}
\caption{One white channel}                                                                 
\label{fig:onewhitechannel}
\vspace{-0.1in}
\end{figure}

In Fig.~\ref{fig:heappercentage}, we investigate the percentage of heterogeneous access points
in 3 white space channels scenario and two white space channels scenario in our algorithm. 
With sufficient white space channels, the result has more heterogeneous access point and there is no 
limitation of the heterogeneous access point deployments. The percentage of heterogeneous access points
is higher than 80\%. In two white space channels scenario, at the beginning the number of hetergeneous
access point is restricted by the interference range. As the population increase, no more heterogeneous
access points could be added in the target area, the carrier have to add more WiFi access point which 
makes the heterogeneous points percentage decrease. When the population distribution reach the threshold 
make the capacity constraint restrict the service area of heterogeneous access point, the percentage of 
heterogeneous access points increase. When the service area continue to shrink to be the same as the WiFi 
access point, this heterogeneous access points have more capacity, the algorithm will make all the access 
points become heterogeneous. 
%In the white space channel shortage scenario (one or two white space channels), 
%white space channel becomes
%the inside service area of heterogeneous access point to explore spatial reuse.



\begin{figure}
%\vspace{-0.0in}
\centering
\includegraphics[width=74mm]{figures/percentage}
\vspace{-0.1in}
\caption{Percentage of Heterogeneous Access Points}                                                                 
\label{fig:heappercentage}
\vspace{-0.1in}
\end{figure}

% sum
As shown in the numerical simulation, the heterogeneous access point could balance
the spatial reuse and large service area to adapt multiple traffic demand scenarios.
The simulation results show the combination of white space bands and WiFi bands could 
reduce the cost for network deployment significantly.






