\section{Path Analysis with Diverse Propagation}
\label{sec:wmalgorithms}


In this section, we discuss the influence of diverse propagation
characteristics of the wide range of carrier frequencies of
white space and WiFi bands. We then introduce two heuristic
algorithms for channel assignment in WhiteMesh networks.
%According to the analysis, we develop two algorithms for ~\emph{Channel Assignment} in multi-band multi-radio scenario.

% PEN part 
\subsection{Path Interference Induced on the Network}
\label{subsec:PEN}
% Talk about the network efficiency for multiband multihop mixed hop

In WhiteMesh networks, multihop paths can be intermixed with WiFi and 
white space bands.  To consider which combination is better, we consider
which band choices reduce the number of hops along a path and the 
aggregate level of interference that hop-by-hop path choices have
on the network (i.e., Path Interference induced on the Network).

%network, the definition of a link is a wireless channel from one node to another, a path is a combination of links connecting two nodes.
%In ~\emph{Multiband Multiradio Network}, 
%a multihop path could be mixed with higher frequency links have less interference range and lower frequency links have less hop count. This is a significant difference from previous ~\emph{Multi-Channel Multi-Radio} work. 
%A key issue of path selection in multi-band network is to answer which link combination is better.

%To discuss this problem, we pick up a multihop path from wireless mesh network and analyze its performance. In wireless mesh network, generally a path would have a bottleneck in the link closest to gateway.
%When a mesh network was built with gateway placement, constructor should considered load-aware demand of mesh nodes and mesh node population. 
%Under this assumption, all the nodes in the path equally share the time of the common links. 
Due to random access, mesh nodes closer to the gateway generally achieve
greater levels of throughput at sufficiently high offered loads. To combat such
starvation effects, we treat each flow with equal priority in the network when
assigning channels. In particular, all nodes along a particular path have equal 
time shares for contending links (i.e., intra-path interference). At the beginning of 
a particular channel assignment scheme, assume that $h$ mesh nodes are demanding
traffic from each hop of an $h$-hop path to the gateway. If each link along the 
path uses orthogonal channels, then each link could be active simultaneously. 
Consider if each node along the path had traffic demand $T_d$, and the bottleneck 
link along the path were closest to the gateway. Then, the total traffic along 
the path $h \cdot T_d$ must be less than the bottleneck link's capacity $\gamma$. 
In such a scenario, the $h$-hop mesh node would achieve the minimum served demand,
which we call the network efficiency. In general, the active time per link for an
$h$-hop mesh node can be represented by $1,\frac{h-1}{h},\frac{h-2}{h}\cdots \frac{1}{h}$.
The summation of all active times for each mesh node along the path is considered the
intra-path network cost.

%The first link close to gateway in the $h$ hop path woule be active for the whole time unit, the sencond would be $\frac{h-1}{h}$, and so on.
%Then the acitve time in a time unit of each link in the path can be represented as $1,\frac{h-1}{h},\frac{h-2}{h}\cdots \frac{1}{h}$. 
%The summation of each link active time in the path is counted as total cost time of network.
%\begin{equation}
%\label{eq:intrapath}
%\begin{split}
%E_{Intra-Path}=\frac{Path\ Active\ Time}{Network\ Time}\\
%E_{Intra-Path}=\frac{1}{2}+\frac{1}{2\cdot h}
%\end{split}
%\end{equation}
%As hop count increase, the ~\emph{Intra-Path} will decrease till the lower bound $\frac{1}{2}$. With routing protocol which is out of this work, the delay increase too.
%More interference comes from \emph{Inter-Path}, which represent interference with links out of the path. 
Considering only intra-path interference, using lower carrier frequencies allows a
reduction in hop count and increases the network efficiency of each mesh node along
the $h$-hop path. However, a lower carrier frequency will induce greater interference
to other paths to the gateway (i.e., inter-path interference). 
Fig.~\ref{fig:interferencerange} depicts such an example where
links in different bands are represented by circles for 450 MHz, rectangles for
2.4 GHz, and triangles for the nodes which can choose between the two.
Nodes $A$ and $C$ could be connected through two 2.4-GHz links or a single 450-MHz link.
With 2.4 GHz, the interfering distance will be less than using 450 MHz. For example, only 
link $D,E$ will suffer from interference, whereas $H,I$ would not. However, with 450 MHz,
link $A,C$ would interfere with links $F,G$, $M,L$, and $K,J$. At each time unit, the number of
links interfering with the active links along a path would be the inter-path network cost.
%\begin{figure}
%%\vspace{-0.0in}
%\centering
%\includegraphics[width=74mm]{figures/networkefficiency}
%\vspace{-0.1in}
%\caption{Path Efficiency Introduction Solid Wire notes 2.4GHz link, Dashed line notes 900MHz}
%\label{fig:networkefficiency}
%%\vspace{-0.0in}
%\end{figure}

%To combine intra-path and inter-path interference, we define each unit of time a link 
%is counted as a unit of network time. 
When an $h$-hop flow is transmitted $T_d$ to a destination node, it prevents 
activity on a number of links in the same band via the protocol model. 
The active time on a single link can be noted as 
$\frac{T}{\gamma_h}$. 
An interfering link from the conflict matrix $F$ counts as $I_h$ per unit time
and contributes to the network cost as:
$\frac{hT}{\gamma_1}\cdot I_1 + \frac{(h-1)T}{\gamma_2}\cdot I_2 \cdots \frac{T}{\gamma_h}\cdot I_h$.
Then, the traffic transmitted in a unit of network cost for the $h$-hop node is:
\begin{equation}
\label{eq:originpen}
E_{\eta}=\frac{T}{\sum_{i \in h}\frac{(h-i+1)\cdot T}{\gamma_i}\cdot I_i }
\end{equation}
%With the protocol model used, if links exist, then they have the same capacity $\gamma_1=\gamma_2 \cdots =\gamma_h=c$. 
Using network efficiency, the equation simplifies to:
\begin{equation}
\label{eq:pen}
E_{\eta}=\frac{\gamma}{\sum_{i \in h} (h-i+1)\cdot I_i}
\end{equation}

The meaning of the network efficiency is the amount of traffic that could be 
offered on a path per unit time. With multiple channels from the same band,
$I_i$ will not change due to the common communication range. With multiple
bands, $I_i$ depends on the band choice.  
%In multichannel scenario, the links will have a common communication range, the $I_i$ will not change according to bands, this parameter equals to the conflic graph in many multichannel works~\cite{jain2005impact}. 
%Since we count only one channel not multiple possible links, the parameter also could be seen as an extention of a single link's load as defined in ~\cite{raniwala2004centralized}.
This network efficiency jointly considers hop count and interference. We define
the Path Interference induced on the Network (PIN) as the denominator of Eq.~\ref{eq:pen},
which represents the sum of all interfered links in the network by a given path. We
use PIN to formulate both of our heuristic algorithms as to how to assign channels
across WiFi and white space bands.
%Then we have to find an answer when a lower white space band is better to be used in a path.
To determine when the lower carrier frequency will be better than two or more links at a
higher carrier frequency, we consider the average interference $\bar{I}$ of a given path
at the higher frequency.  The problem could be formulated as:
\begin{equation}
\label{eq:benefit}
\frac{\gamma}{\frac{h(h-1)}{2}\cdot \bar{I}+I_x} \geq \frac{\gamma}{\frac{h(h+1)}{2}\cdot \bar{I}}
\end{equation}

Here, when $I_x \leq 2\cdot h\bar{I}$, a lower-frequency link could 
be better than two higher-frequency links along the same path. $I_x$ is also a function of hop count 
in Eq.~\ref{eq:pen}. When the hop count is lower, the threshold would be more strict since the
interference would have a greater effect on gateway goodput.
%It matches the intuition the hop order is small, it close to the gateway, it may interference more links so it needs a more strict constraint.
%According to these analysis, to improve the performance of a channel assignment in multi-band multi-radio scenario has two ways. First is to reduce the hop count, second is to reduce the interference among links. And at the same time, we have to trade off between the hop count reduction and single link interference which does not happen in multi-channel multi-radio scenario.





%The discussion in subsection ~\ref{subsec:PEN} provide the methodology to balance hop counts and low frequency long distance links in channel assignment. But the difficulty of channel assignment is that before the process has been done, it could no be evaluated to tell which is better.
%To approach the solution, we propose two local search based heuristic algorithms to adapt the multiband scenario. 

\subsection{Growing Spanning Tree (GST) Algorithm}
\label{subsec:GST}
\begin{algorithm}[t]
\small
\caption{Growing Spanning Tree (GST)}
\label{algorithms:gst}
\begin{algorithmic}[1]
\REQUIRE  ~~\\
	 $M$: The set of mesh nodes\\
	 $G$: The set of gateway nodes\\
	 $C$: Communication graph of potential links among all nodes\\
	 $I$: Interference matrix of all potential links \\
	 $B$: Available frequency bands
\ENSURE ~~\\    
$CA$: Channel Assignment of the Network\\
\STATE Initialize $S_{current}=G$, $N_{served}=\emptyset$, $N_{unserved}=M$,$I_{active}=\emptyset$
\STATE Rank mesh nodes according to physical distance from gateway nodes
\WHILE {$N_{served}=!M$}
\FORALL {$s \in S_{current}$}
	\STATE Find one-hop nodes in $S_{Next}$
	\STATE Sort $S_Next$ according to distance from gateway nodes
	\FORALL {$l \in S_{Next}$}
		\STATE Calculate 1-hop path interference of link $s\rightarrow l$
		\STATE Sort the links according to path interference
		\STATE Assign(s,l) with the least interference link
		\STATE Update $N_{served},N_{unserved}$
		\STATE Update $I_{active}$ from $I$
	\ENDFOR
	\STATE $S_{current}=S_{Next}$
\ENDFOR
\ENDWHILE
\STATE Sort mesh nodes with their hop counts to gateway nodes $N_{sorted}$
\WHILE {Change of Channel Assignment Exists} 
\FORALL {$s \in N_{sorted}$}
	\STATE Traverse all 1-hop arrived nodes have less hop count than node $s$ 
	\STATE Check if these nodes have radio slots for node $s$
	\STATE Sort path through possible nodes with the path interference
	\STATE Choose a new path if it has less interference than the previous one
	\STATE If more than one path has the same interference, choose least-leaved gateway node
\ENDFOR
\ENDWHILE

Output $Channel Assignment$ as Solution
\end{algorithmic}
\end{algorithm}

In a mesh network, gateway nodes tend to be located at the points
of most dense demand~\cite{robinson2008adding, he2008optimizing}.
In the mesh topology, the closer a mesh node is to the gateway,
the more interference it will likely have due to higher demand.
Conversely, edges of the network tend to have more sparse demand,
resulting in less interference. Based on this intuition,
the Growing Spanning Tree (GST) algorithm (described in Alg.~~\ref{algorithms:gst})
assigns channels to have the least resulting interference on the network (PIN) in a
greedy manner. To do so, we first initialize the mesh-node ranking
with respect to the physical distance to all gateway nodes.
We then consider the one-hop nodes from the gateways (based upon
if any carrier frequency of the available bands $B$ is in
communication range of the gateway) with least Path Interference
induced on the Network (PIN) for these available band. This
least-interfering, one-hop node is chosen for channel assignment,
and the network is updated for the next step. We term this Phase~1
of the GST, and it resembles the Breadth First Search Channel
Assignment (BFS-CA)~\cite{ramachandran2006interference}.

In Phase~2 of the GST algorithm, we sort the mesh nodes according
to their hop count.  The algorithm then
traverses all the nodes whose hop count are less than the current node.
If there are available radio slots for the mesh nodes of lower hop
count from the gateway, it is possible to reassign the mesh node
to reduce the hop count.  We rank all possible options with their PIN.
We then choose the lowest one for reassignment of the mesh node. If
there exists new links has the same PIN to two or more gateways, we
consider the total number of nodes connected to each gateway, selecting
the gateway that has fewer connected mesh nodes. Phase~2 process will
iterate until no changes in channel assignment occur or up to the total
number of mesh nodes.

% Need to talk about how to improve the bottle neck links,
%FIXME talk about BFS-CA 

% Talk a little bit about the tree growing and continue to the best path
The GST algorithm greedily assigns a single link to the network (Phase~1) 
and balances the gateway load (Phase~2). 
The breadth first search from Phase~1 for a multiband network has a complexity 
of $O((N_B \cdot N_V)^2)$, where $N_V$ is the number of nodes $V$, $N_B$ is the number of bands, 
and sorting of the mesh nodes would cost $O(N_B \cdot N_V log(N_B \cdot N_V))$. 
Hence, assigning a mesh node takes $O((N_B \cdot N_V)^2)$ time. When there are $N_V$ nodes, 
the complexity of an adjustment iteration is $O(N_B^2 \cdot N_V^3)$.
The total iteration would be less than $N_V$ since we have an upper bound.  
Nonetheless, in our analysis, the complexity does not approach $\frac{N_V}{2}$.
Thus, the complexity of the method would be $O(N_B^2 \cdot N_V^4)$.

\subsection{Band-based Path Selection (BPS) Algorithm}
\label{subsec:BPS}

\begin{algorithm}[t]
    \small
\caption{Band-based Path Selection (BPS)}
\label{algorithms:bps}
\begin{algorithmic}[1]
\REQUIRE  ~~\\
	$M$: The set of mesh nodes\\
	$G$: The set of gateway nodes\\
	$C$: Communication graph of potential links among all nodes\\
	$I$: Interference matrix of all potential links \\
	$B$: Available frequency bands
\ENSURE ~~\\    
$CA$: Channel Assignment of the Network\\
\STATE Rank mesh nodes according to physical distance from gateway nodes
\STATE Initialize $S_{current}=G$, $N_{served}=\emptyset$, $N_{unserved}=M$,$I_{active}=\emptyset$
\WHILE {$N_{served}=!M$}
\STATE Select node with largest distance to gateway nodes
\STATE Find the Adjacency Matrix in different band combinations $A_c$
\FORALL{$A_{i}\in A_c$}
\STATE Find the shortest path $SP_i$ in the mixed adjacency matrix A 
\FORALL{Link $l \in SP_i$ in order from gateway node to mesh node}
\STATE Find the link that has less interference
\STATE If there are links have the same interference, choose higher frequency
\STATE Calculate the path interference of path $SP_i$
\ENDFOR
\STATE Store the shortest path $SP_i$ as $SP$
\ENDFOR
\STATE Assign the path in the Network\\
		\STATE Update $N_{served},N_{unserved}$
		\STATE Update $I_{active}$ from $I$
\ENDWHILE 

Output $CA$ as locally-optimal solution\\
\end{algorithmic}
\end{algorithm}

While the GST algorithm originates from the gateway nodes to the leaf
nodes to assign channels, the Band-based Path Selection (BPS) algorithm
(described in Alg.~\ref{algorithms:bps}) first chooses the mesh node who has the largest
physical distance from the gateway nodes. When a path is constructed for
the mesh node with the greatest distance, all subsequent mesh nodes along
the path are also connected to the gateway. The central concept behind the
BPS algorithm is to improve the worst mesh node performance in a path.
For large-scale mesh networks, it is impractical to traverse all the paths with
different combination of bands from a mesh node to any gateway node. However,
based on the analysis in Section~\ref{subsec:PEN}, if two paths have the same
number of used bands along those paths, then the path with the least hops
is likely to have the greatest performance and is chosen.  Similarly, if
two path have the same path interference, we choose the path who has
higher-frequency links for spatial reuse. Thus, the next step of the
algorithm is to find the shortest path in different band combinations.

Compared to the number of mesh nodes, the amount of channels $N_B$ in
different bands is small. The time complexity of calculation the combination
is $O(2^{N_B})$. Finding the shortest path in Dijkstra algorithm will
cost $O(N_E^2)$ ~\cite{golden1976shortest}, where $N_E$ is the links in the
network, and as a result, the total complexity would be $O(N_E^2\cdot 2^{N_B})$.
The algorithm would then calculate the PIN of the candidate path and select the path
with the least interference induced on the network for the starting mesh node.
After a path is assigned, the algorithm updates the network's channel assignment
with served nodes, activated links, and nodes' radio information. Then,
we assign the next node until all the mesh nodes are assigned channels in the
network.

If all the nodes are connected ($N_E={n \choose 2}$ which is $O(N_V^2)$), 
then the complexity of assigning a channel for a mesh node is $O(N_E^2\cdot2^{N_B})$. 
Then, the complexity of assigning a mesh node is $O(N_V^4\cdot2^{N_B})$.
To assign {\it all} the nodes in the network, the complexity would 
be $O(N_V^5\cdot2^{N_B})$.


